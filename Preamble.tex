\documentclass[10pt,letterpaper]{article}
\usepackage[letterpaper,margin=1in]{geometry}
\usepackage[utf8]{inputenc}
\usepackage{mdwlist}
\usepackage[T1]{fontenc}
\usepackage{textcomp}
\usepackage{tgpagella}
\pagestyle{empty}
\setlength{\tabcolsep}{0em}

% indentsection style, used for sections that aren't already in lists
% that need indentation to the level of all text in the document
\newenvironment{indentsection}[1]%
{\begin{list}{}%
    {\setlength{\leftmargin}{#1}}%
    \item[]%
}
{\end{list}}

% opposite of above; bump a section back toward the left margin
\newenvironment{unindentsection}[1]%
{\begin{list}{}%
    {\setlength{\leftmargin}{-0.5#1}}%
    \item[]%
}
{\end{list}}

% format two pieces of text, one left aligned and one right aligned
\newcommand{\headerrow}[2]
{\begin{tabular*}{\linewidth}{l@{\extracolsep{\fill}}r}
    #1 &
    #2 \\
\end{tabular*}}

% make "C++" look pretty when used in text by touching up the plus signs
\newcommand{\CPP}
{C\nolinebreak[4]\hspace{-.05em}\raisebox{.22ex}{\footnotesize\bf ++}}

% make resume section formatting
\newcommand{\makeresumesection}[3]
{\hrule
\vspace{-0.4em}
\subsection*{#1}
#2
\vspace{#3}}

\newcommand{\objective}[1]
{\begin{indentsection}{\parindent}
  \begin{description*}
    \item #1
  \end{description*}
\end{indentsection}}

\newcommand{\skills}[3]
{\begin{indentsection}{\parindent}
  \hyphenpenalty=1000
  \begin{description*}
    \item[Proficient Languages:]
    #1
    \item[Familiar Languages:]
    #2
    \item[System Administration Technologies:]
    #3
  \end{description*}
  \end{indentsection}}

\newcommand{\education}[4]
{\begin{itemize}
  \parskip=0.1em
  \item
  \headerrow
    {\textbf{#1}}
    {\textbf{#2}}
  \\
  \headerrow
    {\emph{#3}}
    {\emph{#4}}
\end{itemize}}

\newcommand{\experience}[5]
{\item
    \headerrow
        {\textbf{#1}}
        {\textbf{#2}}
    \\
    \headerrow
        {\emph{#3}}
        {\emph{#4}}
    \begin{itemize*}
    \item #5
    \end{itemize*}}

\newcommand{\honors}[3]
{\item
  \headerrow
    {\textbf{#1}}
    {\textbf{#2}}
    \\
    #3}

\newcommand{\projects}[3]
{\item
  \headerrow
    {\textbf{#1}}
    {\textbf{#2}}
    \\
    #3}

\newcommand{\myreference}[4]
{\item
\headerrow
  {\textbf{#1}}
  {#2}
\\
\headerrow
  {\textbf{#3}}
  {#4}}
